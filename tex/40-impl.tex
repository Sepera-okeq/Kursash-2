\chapter{Стек технологий и решений}
\label{cha:impl}

Для разработки будем использовать данный стек как основу:

\begin{itemize}
    \item Операционная система: Астра Линукс - это российская операционная система на базе GNU / Linux. Астра Линукс поставляется с множеством компонентов, которые поддерживают важные функции, как ядра, системные утилиты и GUI. \cite{dev:astra_linux}
    \item Компилятор: LCC 1.25.20 - это локальный C-компилятор для процессоров e2k \cite{dev:elbrus_lcc}. Это портативный компилятор для языка C с открытым исходным кодом, который поддерживается на многих архитектурах и системах.
    \item Язык программирования: Python 3.10 - это последняя версия популярного языка программирования с открытым исходным кодом. Python известен его читаемостью и лаконичностью, что делает его идеальным для разработки и автоматизации.
    \item Nginx - это мощный веб-сервер с открытым исходным кодом (и обратным прокси), который известен своей скоростью и гибкостью. В данном применении, он может быть использован для обслуживания статического содержимого, балансировки нагрузки, кэширования и т.д.
    \item Эльбрус - это серия высокопроизводительных процессоров разработки российской корпорации "МЦСТ". \cite{dev:elbrus_cpu}
    \item СУБД: PostgreSQL Pro.
    \item Инструменты контейнеризации: Такие инструменты как Docker могут понадобиться для контейнеризации нашей системы и управления ими с целью упрощения развертывания, масштабирования и работы в рамках микросервисной архитектуры.

\Abbrev{СУБД}{Система управления базами данных}

\end{itemize}

На стороне серверной части будет использоваться фреймворк Flask, для упрощения и ускорения времени разработки.

%\begin{listing}[H]
%\cfile{../src/test.c}
%\caption{Пример — test.c} 
%\end{listing}
%\label{lst:c}


%%% Local Variables:
%%% mode: latex
%%% TeX-master: "rpz"
%%% End:
