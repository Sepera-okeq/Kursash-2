\chapter{  Исследование и анализ предметной области}

Для создания собственной системы дистрибуции необходимо изучить ее основные характеристики и ознакомиться с существующими на рынке решениями.

\section{Система дистрибуции модулей и ее смысл}

Система дистрибуции программных модулей — это набор инструментов и служб, предназначенный для управления разработкой, развертыванием, обновлением и управлением программными модулями (библиотеками, компонентами, пакетами или даже целыми программами) в программных проектах. Эти системы обычно используются для упрощения и автоматизации процесса управления зависимостями в программном обеспечении, а также для облегчения интеграции и совместимости различных модулей в больших и сложных системах.

Основные функции системы:

\begin{enumerate}
\item cоздание, удаление, обновление модуля;
\item управление зависимостями модулей;
\item версионность модулей;
\item информирование о модуляx;
\end{enumerate}

При этом система представляет собой ключевой инструмент в области разработки программного обеспечения, подстраивающийся под требования поддержки и разработки продукта. Позволяя обеспечивать бесперебойную работу продукта и его своевременное обновление, система дистрибуции вносит большой вклад в его адаптацию к меняющимся технологическим реалиям и потребностям пользователей.

Программные модули, будучи отдельными составляющими программного продукта с конкретными функциями или даже являясь полноценными программами, облегчают процесс разработки и предоставляют удобство в дальнейшем тестировании и сопровождении. Модульность также позволяет многократно использовать эти модули в разных частях программы, повышая эффективность в процессе разработки.

\section{Методы и способы организации систем дистрибуции}

Системы дистрибуции программных модулей, играя свою роль в увеличении эффективности разработки и обеспечении гарантии надежности работы, могут быть организованы разными способами. Эти способы варьируются от простых, таких как ручное копирование, до автоматических систем, которые включают механизмы контроля версий, автоматическую сборку и тестирование.

\section{Обзор существующих систем дистрибуции на рынке}

Сегодняшний рынок предлагает множество систем дистрибуции программных модулей, применяемых в разных областях – от создания игр до высоконагруженных систем. Среди наиболее популярных можно выделить системы, такие как:

\Define{npm}{Node Package Manager ""--- система управления пакетами, которая позволяет разработчикам JavaScript/Node.js устанавливать, делиться и управлять зависимостями (пакетами библиотек или инструментов) в их проектах}

\Define{pip}{ Pip Installs Packages ""--- стандартный менеджер пакетов для языка программирования Python, который позволяет пользователям устанавливать и управлять дополнительными библиотеками и зависимостями, которые не входят в стандартную библиотеку Python}

\Define{CLI} {Сommand Line Interface ""---  способ взаимодействия между человеком и компьютером путём отправки компьютеру команд, представляющих собой последовательность символов}

\begin{enumerate}
\item npm \cite{packages:npm} для JavaScript и Node.js;

npm работает следующим образом \cite{arch:npm_moment}:

Основа всего -- CouchDB: это центральный сервер базы данных, который хорошо подходит для обработки большого количества запросов на чтение и имеет возможность быстрого копирования и восстановления данных. Однако, он сталкивается с проблемами при хранении пакетных tarball-файлов в виде вложений к документам.

Чтобы улучшить эффективность, npm создал базу данных SkimDB, которая состоит только из метаданных. Это ускоряет генерацию представлений данных и упрощает бэкап.

В дополнение к SkimDB базе данных, npm разработал модуль npm-fullfat-registry. Его задача -- возвращать вложения на место после их извлечения модулем SkimDB.

Записи отправляются непосредственно в базу данных SkimDB. При публикации, tarball-файл вложения извлекается, и после подтверждения загрузки в облачное хранилище Manta, он удаляется из базы данных SkimDB.

Чтобы обеспечить обратную совместимость и сохранить раннее функционирование, полные записи (с возвращенными вложениями) затем копируются обратно в изначальную базу данных isaacs.iriscouch.com.

Каждый отдельный основной сервер собственную базу данных, где происходит все записи, и несколько реплик, обслуживающих чтение и запросы GET и HEAD.

Вся эта система защищена сетью доставки контента Fastly, которая автоматически балансирует нагрузку и защищает от большинства интернет-угроз.

В результате такой архитектуры, npm обеспечивает высокую производительность, уменьшает задержки и обеспечивает гибкость и расширяемость для будущего развития.

\item pip \cite{packages:pip} для Python;
\item Maven \cite{packages:maven} для Java;
\item NuGet \cite{packages:nuget} для платформы .NET;
\item RubyGems \cite{packages:rubygems} для Ruby;
\end{enumerate}

Они облегчают процесс установки, обновления и удаления программных модулей. Несмотря на различие в требованиях и особенностях, все эти системы имеют общую цель - облегчить работу разработчика и обеспечивать стабильность работы программного продукта.

\section{Риски, связанные с существующими системами дистрибуции}

Существуют риски в системах дистрибуции программных модулей, на которые нам необходимо обратить внимание. Санкции могут ограничить доступ к некоторым модулям, репозиториям или целым системам распространения модулей, задерживая разработку или даже ставя под угрозу выполнение проекта.

\subsection{Бэкдоры}

\Define{Бэкдор}{это скрытый механизм в программном обеспечении, позволяющий обходить стандартные методы аутентификации для получения несанкционированного доступа к системе или данным. Происхождение слова идет от английского слова backdoor, что буквально переводится как "задняя дверь"}

Бэкдоры, известные также как лазейки в системе безопасности, представляют из себя методы, позволяющие злоумышленникам или потенциально недружественным странам, проникать в защищенную систему или сеть, тем самым создавая серьезную угрозу информационной безопасности. Данный механизм предоставляет доступ к конфиденциальной информации, угрожая нормальной функциональности программного обеспечения, что может вызвать потерю данных или полную неработоспособность системы, обуславливая неудобства для пользователей.

Исходя из возможности исполнения произвольного кода, внедренный злоумышленником бэкдор оказывается в состоянии изменять функциональность важных систем и компонентов, что включает такие бассейны данных как Intel Management Engine (ME) и прочее. Это может позволить атакующему внедрить в систему вредоносный код.

Например, потенциальный нарушитель мог бы использовать и внедрить вредоностные патчи в открытое програмное обеспечение, которое бы распространялось через стандартные системы дистрибьюции, что в итоге достигнет своего разработчика и в дальнейшем - пользователя. 

\Define{CVE}{Common Vulnerabilities and Exposures ""--- это список известных уязвимостей и дефектов безопасности}

Примером может служить бэкдор, разработанный автором по имени JiaT75, в проекте xz (CVE-2024-3094), который включен во множество дистрибутивов Linux и различные программные продукты, в том числе OpenSSH, широко используемый для обеспечения безопасного доступа к системе \cite{risk:xz_backdoor}. 

Данный вредоносный код может использоваться для последующего обновления модуля, делая бэкдор многоразовым или постоянным. Это дает возможность злоумышленникам обновлять модули, добавлять вредоносные элементы, незаметно получать доступ к системам и информации или даже перемещаться по сети, расширяя сферу воздействия.

Таким образом, бэкдоры и возможность исполнения произвольного кода представляют серьезную угрозу, которая может стать условием для сложных атак и нарушения работы инфраструктуры целиком.

\subsection{Увязвимости систем}

Пакеты в популярных системах дистрибуции, таких как npm и pnpm, могут содержать уязвимости, угрожающие всему программному продукту из-за особенности работы систем. 

Нельзя исключать и действия от самого разработчика пакета. Примером может служить умышленное повреждение пакетов kik и leaf-pad, что вызвало проблемы у многих разработчиков программного обеспечения, включая таких как React, Atlas и других \cite{risk:remove-packages}.

Еще одним примером может служить история с пакетом node-ipc от разработчика RIAEvangelist, который привел к распространению вредоносного кода, целевой аудиторией которого стали устройства с IP-адресами, принадлежащими России \cite{risk:node-npc}, \cite{risk:node-npc-2}. 

Это ставит перед нами необходимость регулярного сканирования на наличие уязвимостей и обновления пакетов для поддержания безопасности.

\subsection{Ограничение доступа}

Существует риск того, что разработчики могут закрыть доступ к своим модулям или полностью к системе дистрибуции модулей.

Например, это случилось с системой pnpm, которая используется для разработки JavaScript и Node.js, которая закрыла доступ к системе и получению модулей для России \cite{risk:pnpm_block_1}, \cite{risk:pnpm_block_2}. 

Чтобы минимизировать возможные негативные последствия, необходимо иметь стратегию, которая позволит обеспечить непрерывное функционирование программного обеспечения. В общем, нам необходимо учесть все эти риски и угрозы при разработке системы дистрибуции программных модулей. 

\section{Заключение и направления дальнейшего проектирования}

В результате проведенного анализа было установлено, что система дистрибуции программных модулей важна для успешной и эффективной разработки программного обеспечения. 

В связи с этим определены основные направления проектирования в данной области, которые необходимо принять во внимание при дальнейшей разработке системы дистрибуции программных модулей, а именно:

\begin{enumerate}
\item разработка самодостаточной системы, независимой от зарубежной инфраструктуры, что усилит контроль и увеличит суверенитет в области IT.
\item обеспечение независимость от конкретного языка программирования, позволяющая использовать систему с различными технологиями и платформами.
\item создание средств защиты от вредоносного поведения с использованием продвинутых механизмов безопасности и блокировки.
\end{enumerate}

Важно подчеркнуть значимость применения сертифицированных ФСБ и ФСТЭК решений и рекомендаций, которые соответствуют национальным стандартам безопасности. Использование таких технологий помогает обеспечить высокий уровень защиты данных и операций в системе дистрибуции, а также гарантировать соответствие рекомендациям по работе и защите информации.

Также стоит включить в план разработки меры по регулярному обновлению безопасности и отслеживанию новых угроз, включая БДУ и прочие сервисы обнаружения увязвимостей, что позволит системе оставаться актуальной и надежной при изменяющихся условиях в сфере кибербезопасности.

\Define{БДУ} {Банк данных угроз безопасности информации (ФСТЭК России)}
\Define{ФСБ} {Федеральная служба безопасности Российской Федерации}
\Define{ФСТЭК} {Федеральная служба по техническому и экспортному контролю Российской Федерации}

\Define{БДУ}{Банк данных угроз безопасности информации ""--- представляет собой официальный реестр потенциальных угроз информационной безопасности. Его ведение осуществляется ФСТЭК России в сотрудничестве с ГНИИИ ПТЗИ ФСТЭК России. Основная цель БДУ — систематизация данных об угрозах, критически важных для государственной безопасности и экономики. Реестр включает описание угроз, их источники, потенциальные цели и рекомендации по предотвращению ущерба}

%В \cite{Pup09} указано, что...
%%% Local Variables:
%%% mode: latex
%%% TeX-master: "rpz"
%%% End:
