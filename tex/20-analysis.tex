\chapter{Исследование и анализ предметной области}
\label{cha:analysis}
%
% % В начале раздела  можно напомнить его цель
%
Для того, чтобы разработать собственную систему дистрибуции, необходимо расмотреть что же она собой представляет и какие имеются решения на рынке.

\section{Система дистрибуции модулей и ее смысл}

Система дистрибуции программных модулей — это набор инструментов и служб, предназначенный для управления разработкой, развертыванием, обновлением и управлением программными модулями (библиотеками, компонентами, пакетами или даже целыми программами) в программных проектах. Эти системы обычно используются для упрощения и автоматизации процесса управления зависимостями в программном обеспечении, а также для облегчения интеграции и совместимости различных модулей в больших и сложных системах.

Основные функции системы включают:

\begin{enumerate}
\item Создание, удаление, обновление модуля
\item Управление зависимостями модулей
\item Версионность модулей
\item Информирование о модулях
\end{enumerate}

При этом система представляет собой ключевой инструмент в области разработки программного обеспечения, подстраивающийся под требования поддержки и разработки продукта. Позволяя обеспечивать бесперебойную работу продукта и его своевременное обновление, система дистрибуции вносит большой вклад в его адаптацию к меняющимся технологическим реалиям и потребностям пользователей.

Программные модули, будучи отдельными составляющими программного продукта с конкретными функциями или даже являясь полноценными программами, упакованные в модуль, облегчают процесс разработки и предоставляют удобство в дальнейшем тестировании и сопровождении. Модульность также позволяет многократно использовать эти модули в разных частях программы, повышая эффективность в процессе разработки.

\section{Методы и способы организации систем дистрибуции}

Системы дистрибуции программных модулей, играя свою роль в увеличении эффективности разработки и гарантии надежности работы, могут быть организованы разными способами. Эти способы варьируются от простых, таких как ручное копирование, до автоматических систем, которые включают механизмы контроля версий, автоматическую сборку и тестирование.

\section{Обзор существующих систем дистрибуции на рынке}

Сегодняшний рынок предлагает множество систем дистрибуции программных модулей, применяемых в разных областях – от создания игр до высоконагруженных систем. Некоторые из самых популярных среди них включают системы, такие как:

\Define{npm}{система управления пакетами, которая позволяет разработчикам JavaScript/Node.js устанавливать, делиться и управлять зависимостями (пакетами библиотек или инструментов) в их проектах. npm является сокращением от Node Package Manager}

\Define{pip}{стандартный менеджер пакетов для языка программирования Python, который позволяет пользователям устанавливать и управлять дополнительными библиотеками и зависимостями, которые не входят в стандартную библиотеку Python}

\Abbrev{npm} {Node Package Manager}
\Abbrev{pip} {это рекурсивный акроним для "Pip Installs Packages"}
\Abbrev{CLI} {Сommand Line Interface}

\begin{enumerate}
\item npm \cite{packages:npm} для JavaScript и Node.js,
\item pip \cite{packages:pip} для Python,
\item Maven \cite{packages:maven} для Java,
\item NuGet \cite{packages:nuget} для платформы .NET,
\item RubyGems \cite{packages:rubygems} для Ruby.
\end{enumerate}

Они облегчают процесс установки, обновления и удаления программных модулей. Несмотря на различие в требованиях и особенностях, все эти системы имеют общую цель - облегчить работу разработчика и обеспечивать стабильность работы программного продукта.

\section{Риски, связанные с существующими системами дистрибуции}

Существуют риски в системах дистрибуции программных модулей, на которые нам необходимо обратить внимание. Санкции могут ограничить доступ к некоторым модулям, репозиториям или целым системам распространения модулей, задерживая разработку или даже ставя под угрозу выполнение проекта.

\subsection{Бэкдоры}

\Define{Бэкдор}{это скрытый механизм в программном обеспечении, позволяющий обходить стандартные методы аутентификации для получения несанкционированного доступа к системе или данным. Происхождение слова идет от английского слова backdoor, что буквально переводится как "задняя дверь"}

Бекдоры, которые могут быть внедрены злоумышленниками или недружественными странами, создают угрозу безопасности, позволяя доступ к конфиденциальной информации или нарушая функциональность программного обеспечения.

\subsection{Увязвимости систем}

Кроме того, пакеты в популярных системах дистрибуции, таких как npm и pnpm, могут содержать уязвимости, угрожающие всему программному продукту из-за особенности работы систем. 

При этом, нельзя исключать и действия самого разработчика пакета - как в качестве примера является намеренное повреждение пакета kik и leaf-pad, что породило проблему у многих разработчиков ПО, таких как React, Atlas и другие. \cite{risk:remove-packages}

К этим событиям добавляется история с пакетом node-ipc разработчика RIAEvangelist, который привел распространению вредоностного кода, нацеленного на устройства, имеющие IP-адреса России и Беларуси. \cite{risk:node-npc} \cite{risk:node-npc-2}

Это ставит перед нами необходимость регулярного сканирования на наличие уязвимостей и обновления пакетов для поддержания безопасности.

\subsection{Ограничение доступа}

Также существует риск того, что разработчики могут закрыть доступ к своим модулям или полностью к системе дистрибуции модулей. Чтобы минимизировать возможные негативные последствия, необходимо иметь стратегию, которая позволит обеспечить непрерывное функционирование программного обеспечения. В общем, нам необходимо учесть все эти риски и угрозы при разработке системы дистрибуции программных модулей. 

\section{Заключение и направления дальнейшего развития}

В результате проведенного анализа было установлено, что система дистрибуции программных модулей важна для успешной и эффективной разработки программного обеспечения. 

В связи с этим определены основные направления развития в данной области, которые необходимо принять во внимание при дальнейшей разработке системы дистрибуции программных модулей, а именно:

\begin{enumerate}
\item Создание самодостаточной системы, независимой от зарубежной инфраструктуры, что усилит контроль и увеличит суверенитет в области IT.
\item Независимость от конкретного языка программирования, позволяющая использовать систему с различными технологиями и платформами.
\item Защита от вредоносного поведения с использованием продвинутых механизмов безопасности и блокировки.
\end{enumerate}

Важно подчеркнуть значимость применения сертифицированных ФСБ и ФСТЭК решений, которые соответствуют национальным стандартам безопасности. Использование таких технологий помогает обеспечить высокий уровень защиты данных и операций в системе дистрибуции, а также гарантировать соответствие рекомендациям по работе и защите информации.

Также стоит включить в план разработки меры по регулярному обновлению безопасности и отслеживанию новых угроз, включая БДУ и прочие сервисы обнаружения увязвимостей, что позволит системе оставаться актуальной и надежной при изменяющихся условиях в сфере кибербезопасности.

\Abbrev{БДУ} {Банк данных угроз безопасности информации (ФСТЭК России)}
\Abbrev{ФСБ} {Федеральная служба безопасности Российской Федерации}
\Abbrev{ФСТЭК} {Федеральная служба по техническому и экспортному контролю Российской Федерации}

\Define{БДУ}{Банк данных угроз безопасности информации ""--- представляет собой официальный реестр потенциальных угроз информационной безопасности. Его ведение осуществляется ФСТЭК России в сотрудничестве с ГНИИИ ПТЗИ ФСТЭК России. Основная цель БДУ — систематизация данных об угрозах, критически важных для государственной безопасности и экономики. Реестр включает описание угроз, их источники, потенциальные цели и рекомендации по предотвращению ущерба.}

%В \cite{Pup09} указано, что...
%%% Local Variables:
%%% mode: latex
%%% TeX-master: "rpz"
%%% End:
