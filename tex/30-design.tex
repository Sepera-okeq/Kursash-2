\chapter{Проектирование системы дистрибуции программных модулей}
\label{cha:design}

Проектирование системы дистрибуции программных модулей - это сложная задача, ставка при которой делается не только на функциональность, но и на безопасность. 

\section{Выбор архитектуры системы}

Современная разработка ПО основывается на двух основных типах архитектуры: монолитной и микросервисной. Давайте кратко рассмотрим каждый из этих типов. 

Монолитная архитектура представляет собой классическую модель подхода к программному обеспечению, где используется один самостоятельный модуль, функционирующий отдельно от других приложений. Именно поэтому ее и называют "монолитной". Все фрагменты кода и бизнес-функции в такой архитектуре объединены в одной области. \cite{arch:monovsmicro}

\begin{figure}
  \centering
  \includegraphics[width=.7\textwidth]{graphics/img/mono.png}
  \caption{Пример монолитной архитектуры}
  \label{fig:mono}
\end{figure}


Микросервисная архитектура - это разделение архитектуры на множество независимо функционирующих служб. Каждая из этих служб имеет свою бизнес-логику и систему управления базами данных (СУБД), служащую определенной цели. Оба эти типа архитектур стоит рассмотреть в контексте сравнения.

\begin{figure}
  \centering
  \includegraphics[width=.9\textwidth]{graphics/img/micro.png}
  \caption{Пример микросервисной архитектуры}
  \label{fig:micro}
\end{figure}

Среди недостатков монолитной архитектуры можно выделить плохую масштабируемость, трудность внедрения новых модулей и проблемы при обновлении технологий, так как это может затронуть весь проект, к тому же есть и проблемы с отказоустойчивостью и трудностью в горизонтальной масштабируемости. Однако она проще в развертывании, разработке, тестировании и отладке, чем микросервисная. \cite{arch:monovsmicro}

Микросервисная архитектура обладает гибкостью: каждая команда может разрабатывать и разворачивать свой сервис независимо, что ускоряет сроки реализации проектов и позволяет выбирать технологии согласно предпочтениям команды. Отказоустойчивость в данном случае - тоже немаловажный плюс, поскольку при сбое одного микросервиса функциональность приложения страдает лишь частично. Недостатками же можно считать большие затраты при разрастании приложения и сложность отладки и координации между разными командами.

Текущий проект работы подразумевает под собой максимальную нагрузку, портируемость и отказоустойчивость. С учетом этого, было решено использовать микросервисную архитектуру, да и существующие системы неспроста являются микросервисными, так как высокие требования делают зависимость от более гибкой системы.

\section{Функциональные возможности системы}

Выробатанными требованиями мы приходим к функциональным возможностям проектируемой системы:

\subsection{Серверная часть системы}

Серверная часть системы должна обладать данными функциональными возможностями, такими как:

\begin{enumerate}
    \item Загрузка на сервер, отправка пакета (модуля) с сервера.
    \item Управление зависимостями модулей.
    \item Система версий модулей.
    \item Информирование о модулях и их безопасности.
    \item Аутефикация и авторизация на основе JWT.
    \item Реализация открытого API.
    \item Система блокировки и защиты особо критических модулей.
\end{enumerate}

\Abbrev{API}{Application Programming Interface}

\Define{API}{Application Programming Interface ""--- представляет собой набор правил и инструкций, согласно которым различные программы и сервисы могут общаться между собой. Эти правила определяют, как данные и функциональность могут быть переданы от одной программы к другой, как они могут взаимодействовать и обмениваться информацией}

\subsection{Клиентская часть системы}

Клиентская часть системы должна обладать данными функциональными возможностями, такими как:

\begin{enumerate}
    \item Простой и понятный интерфейс через CLI.
    \item Конфигурирование клиентской части.
    \item Управление установленными модулями, включая загрузку, обновление и удаление.
    \item Информирование о модулях и их безопасности.
\end{enumerate}

\section{Микросервис авторизации}

В предложенной работе необходимо разработать микросервис авторизации для системы. Он будет отвечать за следующие функции: 

\begin{enumerate}
    \item регистрацию пользователя.
    \item аутентификацию и авторизацию пользователя.
    \item система сессий через JWT.
    \item система прав у пользователей.
\end{enumerate}

\begin{figure}
  \centering
  \includegraphics[width=0.8\textwidth]{graphics/img/jwt.png}
  \caption{Функционирование JWT в связке клиент-сервер}
  \label{fig:jwt}
\end{figure}

\Abbrev{JWT}{JSON Web Token}
\Define{JWT}{JSON Web Token ""--- руководствующийся открытым стандартом (RFC 7519), представляет собой универсальное средство формирования токенов доступа, рснованный на надежном и широкораспространенном формате JSON, он стал одним из наиболее эффективных и безопасных способов обеспечения эффективной передачи информации между устройствами и серверами в кодированном виде}

\section{API Gateway}
Для полноценного функционирования и получения доступка к микросервисам из любой точки Интернета, необходима единая точка входа, называемая API Gatawey, в качестве которого будет выступать прокси сервер nginx, с модулем поддержки JWT. \cite{arch:api}
\begin{enumerate}
    \item проксирование на необходимые микросервисы;
    \item балансировка нагрузки на запущенные микросервисы;
    \item проверка и прочие действия с JWT;
    \item реализация всех современных стандартов безопасности (CORS и прочие механизмы);
\end{enumerate}

\Define{nginx}{это HTTP-сервер и обратный прокси-сервер, почтовый прокси-сервер, а также TCP/UDP прокси-сервер общего назначения, изначально написанный Игорем Сысоевым}

\Define{CORS-заголовки}{(Cross-Origin Resource Sharing) ""--- это механизм веб-безопасности, который позволяет браузеру загружать данные из стороннего интернет-источника.}

\Define{API Gatawey}{шлюз для API, который упрощает их управление и делает их доступными для клиентов.}

Таким образом, только API Gateway и микросервис авторизации имеют доступ к единому хранилищу, где располагается секретный ключ, используемый для подписи JWT, что повышает безопасность системы в целом. Для предовращения и снижения риска компроментации, ключ рекомендуется менять раз в месяц. 

\section{Микросервис управления пакетами}

Микросервис управления модулями является основным и включает в себя ряд функций и способностей. 

\subsection{Базовый функционал}
Сервис обеспечивает добавление новых модулей, управление их распределением и поиск по модулям в соответствии с запросами пользователя.

\subsection{Система зависимостей}
Сервис предоставляет возможность модулям иметь систему зависимостей, что позволяет создавать модульные приложения и пакеты.

\subsection{Блокировка, поиск уязвимостей и удаление модулей}
Ключевую роль в функционале микросервиса играет возможность блокировки модулей, что является важным средством контроля доступа. Кроме того, микросервис может производить поиск и устранение уязвимостей в модулях, что особенно важно с точки зрения безопасности системы.

\subsection{Информирование}
Функция информирования обеспечивает прозрачность и контроль над процессами управления модулями. С помощью неё пользователи могут получать обновленную информацию о статусе операций и о состоянии модулей.

\section{CDN}

\Abbrev{CDN}{Content Delivery Network}
\Define{CDN}{Content Delivery Network ""--- это сеть распределенных серверов, которые эффективно передают контент пользователям, основываясь на их географическом положении, источнике контента и сервере с оригинальным содержимым}

В контексте проектирования системы, CDN обеспечивает надежную, высокоскоростную и эффективную доставку модулей от сервера к клиенту. Благодаря CDN, пакеты кода или модулей могут быть быстро и эффективно переданы пользователям независимо от их географического расположения.

\begin{figure}
  \centering
  \includegraphics[width=.8\textwidth]{graphics/img/sheme-cdn}
  \caption{Пример визуализации работы CDN}
  \label{fig:mono}
\end{figure}

Когда разработчик выдает команду для установки определенного модуля, клиент обращается к своему реестру (образу базы данных всех доступных модулей), который размещен в CDN.

Работая в совокупности с CDN, система дистрибрюции осуществляет следующие функции:

\begin{enumerate}
    \item Обеспечивает надежную и быструю доставку пакетов на компьютеры разработчиков.
    \item Снижает задержку в сети благодаря географическому размещению серверов CDN ближе к пользователям.
    \item Обеспечивает глобальное масштабирование, так как CDN может эффективно обслуживать тысячи запросов по всей России в разных частях и всему миру.
    \item Увеличивает отказоустойчивость и распределенность системы, поскольку множество точек присутствия CDN могут обслуживать запросы в случае отказа какой-либо из них.
    \item Снижает нагрузку на основные сервисы за счет кеширования.
\end{enumerate}

 При проектировании я решил использовать услуги отечественного провайдера CDN компании Selectel. \cite{cdn:selectel} Выбор данного провайдера был сделан на основе ряда преимуществ, которые он предлагает.

Selectel предоставляет надежные CDN-услуги, что значительно ускоряет загрузку контента, где бы пользователь не находился. Это особенно важно для нашего проекта, так как ориентир на географически распределенную аудиторию и нуждаемся в том, чтобы модули доставлялись быстро и надежно.

Компания обеспечивает высокую доступность пакетов из-за использования глобальной сети серверов внутри России и по всему миру.

При этом зона собственного покрытия CDN Selectel включает себя такие города \cite{cdn:selectel}, как:
\begin{itemize}
    \item Россия: Барнаул, Владивосток, Екатеринбург, Иркутск, Казань, Кемерово, Кизляр, Краснодар, Красноярск, Москва, Новосибирск, Орел, Ростов-на-Дону, Санкт-Петербург, Самара, Симферополь, Уфа, Хабаровск, Южно-Сахалинск
    \item Азия: Алматы, Бишкек, Гонконг, Сингапур, Ташкент
    \item Америка: Ашберн, Сан-Паулу
    \item Европа: Амстердам, Минск, Сухум, Франкфурт
\end{itemize}

\begin{figure}
  \centering
  \includegraphics[width=.8\textwidth]{graphics/img/cdn.scheme.5Y4Ydf}
  \caption{Пример визуализации работы CDN в рамках России}
  \label{fig:cdn_russia}
\end{figure}

Это увеличивает отказоустойчивость проекта и позволяет нам быть уверенными в постоянной доступности модулей для пользователей.

Серверы Selectel обладают высокой пропускной способностью, что среднее время отлика составляет 30 милисекунд \cite{cdn:selectel}, что обеспечивает максимальную скорость передачи данных. Это особенно ценно для системы модульного менеджера, поскольку она имеет дело с большим количеством пакетов, которые должны быть быстро доставлены пользователям.

Кроме того, Selectel уделяет особое внимание вопросам безопасности и защиты данных. Наши пакеты и пользовательские данные хранятся с учетом последних стандартов безопасности, и мы можем быть уверенны в их надежности и защищенности.

При работе с отечественным провайдером у нас также есть возможность получить более оперативную техническую поддержку и консультации по связанным вопросам, что значительно упрощает процесс взаимодействия и решения возникающих вопросов.

В целом, CDN как он встроен в систему, является одним из ключевых факторов обеспечения эффективной и надежной доставки пакетов и модулей, а использование услуг CDN от Selectel, обеспечивает надежность, высокую скорость доставки пакетов, а также высокую степень безопасности данных, что ускоряет время развертывания и обновления проектов, но и снижает время простоя, повышая общую производительность и эффективность процесса разработки ПО.

\section{Оборудование и ОС как часть системы}

Учитывая связанные риски и выроботонные с этим направления, при проектировании необходимо создать решение, совместимое с российским процессором компании МЦСТ, именованный «Эльбрус-8СВ». \cite{dev:elbrus_cpu}

Эльбрус-8СВ - является восьмиядерным промышленным процессором на базе архитектуры Эльбрус (e2k) с использованием длинного командного слова (VLIW), разработанным в России. Он создан для работы в 64-разрядной вычислительной системе и может обрабатывать до 288 гигафлопс двойной точности в одноядерном режиме при частоте 1.5 ГГц, будучи произведенным по техническим нормам в 28 нанометров. 

\Define{Гигафлопс}{Индикатор, определяющий скорость работы суперкомпьютера. 1 г= 109 aial/s, т. е. суперкомпьютер 1 сек.в 1 млрд. выполняет арифметические и логические операции.}

\Abbrev{VLIW}{Very Long Instruction Word}
\Define{VLIW}{Very Long Instruction Word ""--- архитектура процессоров с несколькими вычислительными устройствами. Характеризуется тем, что одна инструкция процессора содержит несколько операций, которые должны выполняться параллельно. Фактически это «видимое программисту» микропрограммное управление, когда машинный код представляет собой лишь немного свёрнутый микрокод для непосредственного управления аппаратурой}

Ключевым преимуществом Эльбруса является его отечественное происхождение, что даёт возможность избегать определенных проблем и рисков, связанных с использованием процессоров Intel и AMD. При этом, параметры процессора позволяют полностью закрыть вопрос в потребностях сервисов.

\begin{figure}
  \centering
  \includegraphics[width=.6\textwidth]{graphics/img/elbrus}
  \caption{Внешний вид процессора компании МЦСТ - Эльбрус-8СВ}
  \label{fig:elbrus}
\end{figure}

В свете недавних проблем с безопасностью в процессорах Intel и AMD, таких как утечка данных через атаки Spectre и Meltdown, проблемы с микрокодом и другие, использование процессора Эльбрус представляет собой весьма привлекательное решение. Он обладает потенциалом минимизировать тактические и стратегические риски связанные с используемым железом, включая риски иностранных поставок, уязвимости безопасности и ответственность за продукт.

Атаки Spectre и Meltdown стали известны в 2018 году и поставили под угрозу безопасность процессоров Intel, ARM, AMD и некоторых других. Эти атаки используют уязвимости в методах оптимизации работы процессора — так называемой спекулятивной и предсказательной вычислительной системе.

Meltdown и Spectre эксплуатируют «спекулятивное выполнение» \cite{risk:spectual_hack} - функцию, при которой процессор предварительно вырабатывает инструкции, которые могут понадобиться в дальнейшем, чтобы ускорить выполнение кода. Это делает системы уязвимыми, поскольку атакующие могут гипотетически пройти по этим предварительным вычислениями и получить доступ к конфиденциальной информации, которую обычно не должны были бы видеть.

Внедрение исправлений для этих уязвимостей в микрокод процессоров Intel и AMD столкнулось с рядом проблем. Например, первоначальные исправления от Intel вызвали проблемы с перезагрузкой и привели к снижению производительности у некоторых пользователей. Затем Intel выпустил обновленные патчи, чтобы решить эти проблемы. AMD также столкнулась с проблемами, связанными с исправлениями Spectre, но в конечном итоге выпустила серию обновлений микрокода для устранения этих уязвимостей.

\Abbrev{Intel ME}{Intel Management Engine}
При этом, Intel ME, это встроенный модуль микроконтроллера, присутствующий во всех чипсетах Intel с 2006 года, который имеет полный доступ к памяти компьютера, сети и другим частям системы, даже когда компьютер выключен или система спит.

Аналогом Intel Management Engine в мире AMD является AMD Secure Technology, которую ранее называли Secure Processor (или Platform Security Processor, PSP). PSP от AMD также был подвержен уязвимости, схожей у Intel ME \cite{risk:amdpsp}.

Уязвимости Intel ME представляют собой нарушение безопасности из-за незадокументированных возможностей данной части процессора, с возможностью удаленно управлять и исполнять код внутри Intel ME, что и демострируют специалисты компании Positive Technologies \cite{risk:intelme}, показывая возможность выполнения кода даже на выключенном сервере. Эти уязвимости позволяют хакерам возиться с компьютером, обходя встроенные защитные механизмы. По сути, это означает, что злоумышленник может полностью контролировать компьютер с уязвимостью Intel ME, управлять его работой и получать доступ к любым его данным.

Для управления используется AMT, который работает постоянно, жесткий диск — нет, но для изменения кода в  Intel ME жесткого диска не нужен. При этом, обычное ожидание ничем не дает безопасности. При этом у Intel есть технология конфигурации с другого хоста, можно в качестве другого хоста подсунуть взломанный Intel ME, и он заразит соседний сервер в результате, а все сервера «выключены». Аппаратные возможности таких решений настараживают и заставляют отказаться от данных процессоров этих компаний.

Однако российские процессоры Эльбрус не были уязвимы для этих атак. Это объясняется архитектурой и методами оптимизации работы процессора, отличающимися от используемых в Intel и AMD. Конкретно, процессоры «Эльбрус» не используют подобное спекулятивное выполнение и системы управления процессором, а значит, уязвимые атаки не могут быть использованы. \cite{risk:elbrus_no_spectre}

В качестве оптимальной операционной системы, которую можно реализовать на оборудовании Эльбрус с использованием пакетного менеджера, можно рассматривать AstraLinux. Это российский дистрибутив на основе Debian GNU/Linux, который нацелен на создание унифицированной, безопасной и легко управляемой операционной системы, c обеспечением высоким уровенем защиты, что особенно важно при работе с критической инфраструктурой, такие как системы дистрибюции модулей. \cite{dev:astra_linux}


%%% Local Variables:
%%% mode: latex
%%% TeX-master: "rpz"
%%% End:
