\Introduction

В современном мире информационных технологий, беспрерывного технического прогресса и совершенствования программного обеспечения, особенное внимание уделяется системам дистрибуции программных модулей. Эта тема становится все более актуальной из-за возрастающего усложнения архитектуры приложений и необходимости быстрого и безболезненного деплоя (распространения) новых версий.

Децентрализация приложений, переход к микросервисной архитектуре, развитие облачных технологий, прочие факторы (в том числе и геополитические) – все это создает новые вызовы для систем дистрибуции. С учетом вышеуказанных тенденций, системы дистрибуции являются ключевым звеном в обеспечении постоянного обновления ПО, рапространения патчей и исправлений ошибок без простоя системы и значительных затрат времени. 

Целью данной работы является исследование существующих систем дистрибуции, а так же разработка системы дистрибуции программных модулей для программного обеспечения. Это необходимо для обеспечения удобства и эффективности работы с программным обеспечением, позволяет облегчить процесс обновления и масштабирования программного продукта. Данная система представляет из себя информационный ресурс, помогающий организовать и оптимизировать процесс дистрибуции программных модулей для потребителей.

В рамках курсовой работы имеется задачи:

\renewcommand{\labelenumi}{\arabic{enumi})}
\renewcommand{\labelenumii}{\asbuk{enumii})}

\begin{enumerate}
\item исследовать и проанализировать предметную область,
\item спроектировать систему дистрибуции программных модулей,
\item выбрать технологии для разработки системы,
\item разработать систему дистрибуции,
\item провести системное тестирование,
\item проанализировать полученные результаты,
\item оформить текст курсовой работы.
\end{enumerate}
