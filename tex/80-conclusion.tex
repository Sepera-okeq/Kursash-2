\Conclusion % заключение к отчёту

В результате проделанной работы были полученны данные результаты:

\begin{enumerate}
    \item мировой опыт в области систем был изучен, проанализированы преимущества и недостатки, а также потенциальные риски.
    \item определены функциональные требования к создаваемому сервису, на основе которых были спроектированы и реализованы два независимых микросервиса, отвечающих за аутентификацию и управление пакетами.
    \item спроектирован проект, в ходе которого была разработана система распределения модулей.
    \item отбраны российские решения для интеграции во свою систему, чтобы избежать зависимости от иностранной инфраструктуры.
    \item реализовано минималистичное приложение с клиентской стороны в формате CLI, предлагающее простой и удобный интерфейс для управления модулями.
    \item реализовано серверное приложение, предоставляющее возможности управления модулями, включая системы анализа безопасности.
    \item проведена адаптация кода существующих приложений для совместимости с процессорами Эльбрус, учитывая специфические особенности архитектуры e2k.
    \item адаптирован интерпретатор Python для платформы e2k, что позволило использовать для выполнения кода.
    \item проведено тестирование на целевых системах с процессорами Эльбрус для верификации корректности и производительности разработанных решений.
    \item проведена работа по интеграции с отечественными операционными системами, такими как Astra Linux и Elbrus OS, для обеспечения максимальной совместимости и производительности.
\end{enumerate}

\newpage

За время выполнения курсовой работы были реализованы следующие компетенции:

\captionsetup[table]{justification=raggedright,singlelinecheck=false}
\begin{table}[h!]
\centering
\caption{Компетенции при выполнении курсовой работы}
\begin{tabular}{|p{3cm}|p{5cm}|p{7cm}|} 
    \hline
    Шифр \newline компетенции & Расшифровка \newline приобретаемой \newline компетенции & Расшифровка освоения \newline компетенции \\[0.5ex] 
        \hline
        УК-6 & Способен управлять своим временем, выстраивать и реализовывать траекторию саморазвития на основе принципов образования в течение всей жизни &  Процесс реализации системы был разбит по шагам, которые выполнялись в отведенное для них время, включая отдельно поэтапное время на изучение.  \\ \hline
        ПК-4 & Разработка требований и проектирование программного обеспечения & Проанализирована тщательно ситуация на рынке систем с учетом текущей ситуации и выработаны требования к системе. \\  \hline
        ПК-5 & Оценка и выбор варианта архитектуры программного средства & Был проведен изучение существующих решений, а так-же выработаны из требований условия функционирования системы, что в результате превратилось в клиент-серверную, где сервер использует микросервесную архитектуру. \\  \hline
        ПК-6 & Разработка тестовых случаев, проведение тестирования и исследование результатов & Проведен полный комплекс тестов, потвердивших работоспособность всей системы, начиная от проектирования и вплоть до разработки.\\ 
 \hline
\end{tabular}

\label{table:1}
\end{table}

%%% Local Variables: 
%%% mode: latex
%%% TeX-master: "rpz"
%%% End: 
