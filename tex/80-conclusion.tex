\Conclusion % заключение к отчёту

В результате проделанной работы были реализованы все основные поставленные задачи. Был проведен комплекс анализа, спроектирована и разработана система дистрибьюции модулей. Был изучен мировой опыт на рынке систем, проанализированы все за и против, как и все потенциальные риски. Проведен анализ архитектур систем, результатом которой стала микросервесная архитектура. Были отобраны российские решения, чтобы локализовать под собственные решения и не зависеть от зарубежной инфракстуктуры. 

Решение о применении микросервисной архитектуры на серверной стороне было принято из-за ее способности разделять логику приложения на отдельные сервисы, что увеличивает отказоустойчивость и гибкость всей системы. Функциональные требования к разрабатываемому сервису были определены и на их основе было спроектировано и реализовано два независимых микросервиса, отвечающих за аутентификацию и управление пакетами, при этом вход осуществлялся по токенам. 

На стороне клиента было реализовано минималистичное приложение в формате CLI, что не нагружает черезмерным GUI, предоставляя простой интерфейс с удобными командами для управления модулями. Оно может быть использовано при активной разработке различных проектов, начиная от мелких сфер - до геймдева и сложных проектов.



За время выполнения курсовой работы были реализованы следующие компетенции:
\begin{table}[h!]
\centering
\begin{tabular}{|p{3cm}|p{5cm}|p{7cm}|} 
    \hline
    Шифр \newline компетенции & Расшифровка \newline приобретаемой \newline компетенции & Расшифровка освоения \newline компетенции \\[0.5ex] 
        \hline
        УК-6 & Способен управлять своим временем, выстраивать и реализовывать траекторию саморазвития на основе принципов образования в течение всей жизни &  Процесс реализации системы был разбит по шагам, которые выполнялись в отведенное для них время, включая отдельно поэтапное время на изучение.  \\ \hline
        ПК-4 & Разработка требований и проектирование программного обеспечения & Проанализирована тщательно ситуация на рынке систем с учетом текущей ситуации и выработаны требования к системе. \\  \hline
        ПК-5 & Оценка и выбор варианта архитектуры программного средства & Был проведен изучение существующих решений, а так-же выработаны из требований условия функционирования системы, что в результате превратилось в клиент-серверную, где сервер использует микросервесную архитектуру. \\  \hline
        ПК-6 & Разработка тестовых случаев, проведение тестирования и исследование результатов & Проведен полный комплекс тестов, потвердивших работоспособность всей системы, начиная от проектирования и вплоть до разработки.\\ 
 \hline
\end{tabular}
\caption{Компетенции при выполнении курсовой работы}
\label{table:1}
\end{table}

%%% Local Variables: 
%%% mode: latex
%%% TeX-master: "rpz"
%%% End: 
